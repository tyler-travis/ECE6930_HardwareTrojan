%%%%%%%%%%%%%%%%%%%%%%%%%%%%%%%%%%%%%%%%%%%%%%%%%%%%%%%%%%%%%%%%%%%%%%%%%%%%%%%%
%2345678901234567890123456789012345678901234567890123456789012345678901234567890
%        1         2         3         4         5         6         7         8

\documentclass[letterpaper, 10 pt, conference]{ieeeconf}  % Comment this line out if you need a4paper

%\documentclass[a4paper, 10pt, conference]{ieeeconf}      % Use this line for a4 paper

\IEEEoverridecommandlockouts                              % This command is only needed if 
                                                          % you want to use the \thanks command

\overrideIEEEmargins                                      % Needed to meet printer requirements.

% See the \addtolength command later in the file to balance the column lengths
% on the last page of the document

% The following packages can be found on http:\\www.ctan.org
\usepackage{graphicx} % for pdf, bitmapped graphics files
%\usepackage{epsfig} % for postscript graphics files
%\usepackage{mathptmx} % assumes new font selection scheme installed
%\usepackage{times} % assumes new font selection scheme installed
\usepackage{amsmath} % assumes amsmath package installed
\usepackage{amssymb}  % assumes amsmath package installed
\usepackage{multicol}

\title{\LARGE \bf
Hardware Trojan Implemented on a FPGA  
}


\author{Justin Cox and Tyler Travis
\\ \small{Department of Electrical and Computer Engineering}
\\ \small{Utah State University}
\\ \small{Logan, Utah 84322}
\\ \small{email: justin.n.cox@gmail.com, tyler.travis@aggiemail.usu.edu}
}

\usepackage{listings}
\usepackage{color}

\definecolor{dkgreen}{rgb}{0,0.6,0}
\definecolor{gray}{rgb}{0.5,0.5,0.5}
\definecolor{mauve}{rgb}{0.58,0,0.82}

\lstset{frame=none,
  language=C,
  aboveskip=3mm,
  belowskip=3mm,
  showstringspaces=false,
  columns=flexible,
  basicstyle={\small\ttfamily},
  numbers=none,
  numberstyle=\tiny\color{gray},
  keywordstyle=\color{blue},
  commentstyle=\color{dkgreen},
  stringstyle=\color{mauve},
  breaklines=true,
  breakatwhitespace=true,
  tabsize=3
}

\begin{document}



\maketitle
\thispagestyle{empty}
\pagestyle{empty}


%%%%%%%%%%%%%%%%%%%%%%%%%%%%%%%%%%%%%%%%%%%%%%%%%%%%%%%%%%%%%%%%%%%%%%%%%%%%%%%%
\begin{abstract}


\emph{Index Terms}---secret key generation, security, PUF, FPGA.

\end{abstract}

%%%%%%%%%%%%%%%%%%%%%%%%%%%%%%%%%%%%%%%%%%%%%%%%%%%%%%%%%%%%%%%%%%%%%%%%%%%%%%%%
\section{INTRODUCTION}
  

\subsection{Previous Work}



\subsection{Contribution}



%%%%%%%%%%%%%%%%%%%%%%%%%%%%%%%%%%%%%%%%%%%%%%%%%%%%%%%%%%%%%%%%%%%%%%%%%%%%%%%%
\section{OVERVIEW}

%\subsection{Data Encryption Standard}


%%%%%%%%%%%%%%%%%%%%%%%%%%%%%%%%%%%%%%%%%%%%%%%%%%%%%%%%%%%%%%%%%%%%%%%%%%%%%%%%
\section{8051 IP CORE}

%\begin{figure}[thpb]
%	\centering
%	\includegraphics[scale=.50]{DAPUF}
%   \caption{3-1 DAPUF figure taken from [1].}
%\end{figure}


\section{TROJAN DESIGN}

\subsection{Taxonomy}
  
\section{EXPERIMENTAL RESULTS}

The trojan worked as expected once everything was loaded onto the board. The spacing between the frames from the varying stop bits caused the bits for the key to be leaked out over 8 plain text inputs, with the first one being the triggering plaintext of "pwnage" or 0x7077.6e61.6765 in hex. The only part left to the person leaking the key is to brute force the remaining 8 bits left off by completed ciphertext outputs on the TX line of the FPGA. 

\subsection{Size}
 
The size of this trogan compared others is realitiviy small compared to the other trojans in the class. With only a 6.2\% increase in look up tables (LUTs) at the worst, there isn't a lot of extra area used. Refer to Table xx for more information.

 \begin{center}
    \begin{tabular}{| l | l | l | l | l |}
        \hline
        & Original & With Trojan & Difference & Percentage \\ \hline
        LUTs & 835 & 887 & 52 & 6.2\% \\ \hline
        Slices & 562 & 587 & 25 & 4.4\% \\ \hline
        Flip Flops & 451 & 467 & 16 & 3.5\% \\
        \hline
    \end{tabular}
\end{center}

\subsection{Power}

The power that was consumed by the trojan was a little bit more than other trojans in the class. This is probably due to the size of some of the components in the design.

 \begin{center}
    \begin{tabular}{| l | l | l | l | l |}
        \hline
        & Original & With Trojan & Difference & Percentage \\ \hline
        Volts (supply) & 5.5 & 5.5 & 0 & 0\% \\ \hline
        Amps (supply) & 0.067 & 0.067 & 0 & 0\% \\ \hline
        Volts (resistor) & 1.72mV & 1.84mV & 0.12mV & 6.9\% \\
        Watts & 7.8mw & 8.4mW & 0.6mW & 7.69\% \\
        \hline
    \end{tabular}
\end{center}

\subsection{Detectability}

The detectability of the trojan is pretty low since the trojan is only activated when a certain event triggers it. Therefore, the output of the device will stay consistent as long as the plaintext trigger isn't entered into the device. Once the trigger has been triggered, the device will leak the key through the next 8 UART frames for the attacker to use.

\subsection{Measurements} 



The figures for the measurements are included in the appendix of this report.

\section{CONCLUSION}


\addtolength{\textheight}{-12cm}   % This command serves to balance the column lengths
                                  % on the last page of the document manually. It shortens
                                  % the textheight of the last page by a suitable amount.
                                  % This command does not take effect until the next page
                                  % so it should come on the page before the last. Make
                                  % sure that you do not shorten the textheight too much.

%%%%%%%%%%%%%%%%%%%%%%%%%%%%%%%%%%%%%%%%%%%%%%%%%%%%%%%%%%%%%%%%%%%%%%%%%%%%%%%%



%%%%%%%%%%%%%%%%%%%%%%%%%%%%%%%%%%%%%%%%%%%%%%%%%%%%%%%%%%%%%%%%%%%%%%%%%%%%%%%%



%%%%%%%%%%%%%%%%%%%%%%%%%%%%%%%%%%%%%%%%%%%%%%%%%%%%%%%%%%%%%%%%%%%%%%%%%%%%%%%%

%\section*{ACKNOWLEDGMENT}

%The author would like to thank his instructor Dr. Rajnikant Sharma %for his help in understanding control concepts.




%%%%%%%%%%%%%%%%%%%%%%%%%%%%%%%%%%%%%%%%%%%%%%%%%%%%%%%%%%%%%%%%%%%%%%%%%%%%%%%%


\begin{thebibliography}{99}

\bibitem{c1} T. Machida, D. Yamamoto, M. Iwamoto, K. Sakiyama. A New Mod of Operation for Arbiter PUF to Improve Uniqueness on FPGA. The University of Electro-Communications. Tokyo, Japan. 2014.
\bibitem{c2} J. Orlin Grabbe. The DES Algorithm Illustrated. \emph{Laissez Faire City Times}, 2006.
\bibitem{c3} T. Machida, D. Yamamoto, M. Iwamoto, K. Sakiyama. Implementation of Double Arbiter PUF and its Performance Evaluation on FPGA. The University of Electro-Communications. Tokyo, Japan. 2015.
\bibitem{c4} M. Rostami, M. Majzoobi, F. Koushanfar, D. Wallach, S. Devadas. Robust and Reverse-Engineering Resilient PUF Authentication and Key-Exchange by Substring Matching. Rice University. Houston, Texas. January. 2014.
\bibitem{c5} Martin Deutschmann. Cryptographic Applications with Physically Unclonable Functions. Alpen-Adria-Universit{\"a}t Klagenfurt. Klagenfurt. November. 2010.

\end{thebibliography}

\end{document}
